\documentclass[12pt,a4paper]{article}
\usepackage[utf8]{inputenc}
\usepackage{amsmath}
\usepackage{graphicx}
\usepackage{cite}

\title{Test Thesis Document}
\author{Test Author}
\date{\today}

\begin{document}

\maketitle

\begin{abstract}
This is a sample thesis document for testing local compilation and CI/CD workflows. It includes basic LaTeX elements such as sections, equations, citations, and tables.
\end{abstract}

\section{Introduction}

This is the first section of the test document. We will demonstrate various LaTeX features including mathematical formulas, citations, and basic formatting.

\subsection{Background}

According to recent research\cite{sample2024}, this field has gained significant importance. Here is an example of a mathematical equation:

\begin{equation}
    E = mc^2
\end{equation}

\begin{equation}
    \int_{0}^{\infty} e^{-x^2} dx = \frac{\sqrt{\pi}}{2}
\end{equation}

\section{Methodology}

This section describes the research methodology. We employed multiple techniques\cite{method2023} to address the problem.

\subsection{Experimental Design}

The experiment includes the following steps:
\begin{itemize}
    \item Data collection
    \item Data preprocessing
    \item Model training
    \item Results evaluation
\end{itemize}

\section{Results}

Table example:

\begin{table}[h]
\centering
\begin{tabular}{|c|c|c|}
\hline
Method & Accuracy & F1-Score \\
\hline
Method A & 0.85 & 0.83 \\
Method B & 0.90 & 0.88 \\
Our Method & \textbf{0.95} & \textbf{0.93} \\
\hline
\end{tabular}
\caption{Performance comparison of different methods}
\label{tab:results}
\end{table}

\section{Conclusion}

This paper proposes a new method and validates its effectiveness through experiments. Future work will further improve this approach\cite{future2024}.

\bibliographystyle{plain}
\bibliography{ref}

\end{document}
